\documentclass{article}
\usepackage{amsmath}
\usepackage{amssymb}
\usepackage{mathrsfs}
\title{MAT 530 Homework Assignment 12}
\author{Gabriel C. Drummond-Cole}
\date{\today}
\setlength{\parindent}{0pt}
\setlength{\parskip}{2ex}
%\addtolength{\hoffset}{-1cm}
%\addtolength{\textwidth}{2cm}
\begin{document}
\maketitle

411.
\\4. The order of an element $a$ of an abelian group $G$ is the smallest positive integer $m$ such that $ma=0,$ if such exists; otherwise $a$ has infinite order.  The order of $a$ equals the order of the subgroup generated by $a.$  We denote the order of $a$ by $|a|.$
\\a. Let $S$ be the elements of finite order in $G.$  We show that $S$ forms a subgroup of $G.$  To do this we show that $S\ne\emptyset$ and is closed under subtraction.
\\$|0|=1$ because $1\cdot0=0$ so $0\in S$ and $S$ is not empty.
\\Let $a,b\in S.$  Then $0<|a|,|b|<\infty.$  So $(|a||b|)(a-b)=|a||b|a-|a||b|b$ since $S$ is abelian.  This is $|b|(|a|a)-|a|(|b|b)=|b|0-|a|0=0.$  So $|a-b|\le |a||b|$ which implies $a-b\in S.$  So $S$ is nonempty and closed under subtraction, hence a subgroup.

b. We show that if $G$ is free abelian, it has no elements of finite order.
\\ This is trivially false as stated since $|0|=1$ and every group contains the identity.  We show that $G$ contains no other elements of finite order.
\\
Say $G$ is free abelian.  Then it is generated by an indexed family of elements $\{a_j\}$ where the subgroup $G_j$ generated by each $a_j$ is infinite cyclic and $G$ is the direct sum of the $G_j.$
\\
Now say $a\in G$ has finite order.  Then $a$ can be written uniquely as a sum of elements from each $G_j,$ each of which is just an integer multiple of $a_j,$ finitely many of which are nonzero.  So $a=n_1a_{j_1}+\cdots+n_ka_{j_k}.$  Then $0=|a|a=|a|n_1a_{j_1}+\cdots+|a|n_ka_{j_k}.$  Since 0 can be written as $\sum 0a_j$ in $G$ and this expression is by definition unique, we must have $|a|n_i=0$ for each $1\le i\le k.$  Since $|a|\ne 0$ this implies each $n_i=0$ so that $a=0a_{j_1}+\cdots+0a_{j_k}=0.$  So the only element in $G$ of finite order is the identity.

c. We show that $\mathbb{Q}_+$ has no element of finite order but is not free abelian.
\\
Again, this is trivially false since $0$ has order 1.  Any other element, however, can be written $p/q$ with $p$ and $q$ nonzero integers, and $n(p/q)=np/q$ is zero for no positive integer $n.$
\\
Now say $\mathbb{Q}_+$ were free abelian.  Then it would be generated by the indexed family of elements $\{a_j\}.$
\\
Certainly this family is nonempty because $\mathbb{Q}_+$ is not the trivial group.   Let $a\in\{a_j\}.$ Then $\frac{1}{2}a\in\mathbb{Q}_+$ so $\frac{1}{2}a=n_1a_{j_1}+\cdots+n_ka_{j_k}.$
\\
Then $a=2\frac{1}{2}a=2n_1a_{j_1}+\cdots+2n_ka_{j_k}.$  Since $a$ can be expressed as $1a$ and this expression must be unique, we must have $a_{j_i}=a$ for some $1\le i\le k$ and $2n_{i'}=0$ for $i'\ne i.$
\\
Then $n_{i'}=0$ and $\frac{1}{2}a=n_ia$ for some positive integer $n_i.$  Then $1a=a=2n_ia$ so that $1-2n_i=0.$  Then $n_i=1/2$ but is an integer, a contradiction.

421.
\\2. Let $G=G_1*G_2$ where $G_1$ and $G_2$ are nontrivial groups.
\\a. We show $G$ is not abelian.
\\Let $g\in G_1$ and $h\in G_2$ be nontrivial.  Then $(g,h,g^{-1},h^{-1})$ is a nonempty reduced word in $G$ so is not the identity.  If $G$ were abelian we would have $(g,h,g^{-1},h^{-1})=(g)(h)(g^{-1})(h^{-1})=(g)(g^{-1})(h)(h^{-1})=(g,g^{-1},h,h^{-1})=(gg^{-1},hh^{-1})=(1,1)=(),$ a contradiction.

b. For $x\in G,$ we define the length of $x$ to be the length of the unique reduced word in the elements of $G_1$ and $G_2$ that represents $x.$  We show first that if $x$ has even length at least 2, it does not have finite order.
\\
Let the length of $x$ be $2n.$  In order that a word representation of $x$ be reduced, it must not have adjacent elements of $G_1$ or $G_2$ in it.  Then it must alternate elements of $G_1$ and $G_2$ so that without loss of generality we can assume $x=(g_1,h_1,\ldots,g_n,h_n),$ where $g_i\in G_1$ and $h_i\in G_2.$
\\
Say that the length of $x^m$ is $2mn,$ that the first element in the reduced word representation of $x^m$ is $g_1,$ and that the last is $h_n.$  This is certainly true for $m=1.$  Then $x^{m+1}=xx^m=(g_1,\ldots,h_n)(g_1,\ldots, h_n)=(g_1,\ldots,h_n,g_1,\ldots,h_n).$  This is a reduced word since its factors are and we have $h_n\in G_2,g_1\in G_1.$  Then the length of $x^{m+1}$ is $2n+2mn=2(m+1)n$ and the inductive premises about $g_1$ and $h_n$ are true for $m+1.$
\\
Then this is true for every positive integer so that in particular, the length of $x^m$ is not 0 and thus $x^m$ is not the identity.

Now we show that if $x$ has odd length, then $x$ is conjugate to an element of shorter length.
\\
Again, this statement is untrue as written.  If $x$ has length 1 then the only element of shorter length is the identity.  But if $x$ is conjugate to the identity then $()=(y,y^{-1})=(y,1,y^{-1})=x$ and $x$ has length 0, a contradiction.
\\
But say $x$ has odd length at least 3.  Again, in order that $x$ be reduced, it must alternate elements of $G_1$ and $G_2$ so that without loss of generality we can assume $x=(g_1,h_1,\ldots, g_n,h_n,g_{n+1}),$ where $g_i\in G_1$ and $h_i\in G_2.$
\\
Then $g_1^{-1}xg_1$ is a conjugate of $x$ and is equal to $(g_1^{-1},g_1,h_1,\ldots,h_n,g_{n+1},g_1)=(g_1^{-1}g_1,h_1,\ldots,h_n,g_{n+1}g_1)=(1,h_1,\ldots,h_n,g_{n+1}g_1)=(h_1,\ldots,h_n,g_{n+1}g_1).$  The length of this word is $2n.$  Then the reduced form of $g_1^{-1}xg_1$ can be this long at longest, so that its length is at most $2n<2n+1.$

c.  We show that the only elements of $G$ with finite order are the elements of $G_1$ and $G_2$ with finite order and their conjugates.
\\
Before we begin, we show some basic properties of conjugates.

Being conjugates is an equivalence relation.
\\$x=1x1$ so the relation is reflexive.
\\If $x=gyg^{-1}$ then $g^{-1}xg=g^{-1}gyg^{-1}g=y$ and the relation is symmetric.
\\If $x=gyg^{-1}$ and $y=hzh^{-1}$ then $x=ghzh{-1}g^{-1}=(gh)z(gh)^{-1}$ and the relation is transitive.

Now suppose $x$ and $y$ are conjugates so that $x=gyg^{-1}.$  Suppose $x^m=gy^mg^{-1}.$  This is true for $m=1.$  Then $x^{m+1}=xx^m=gyg^{-1}gy^mg^{-1}=gyy^mg^{-1}=gy^{m+1}g^{-1}$ and the statement is inductively true for every positive integer $m.$

Now say $x\in G$ is a conjugate of an element $y$ of $G_1$ or $G_2$ of finite order.  Since conjugation is reflexive, this case includes all elements of $G_1$ and $G_2$ of finite order.  Then $x=gyg^{-1}$ for some $g\in G.$
\\
So $x^{|y|}=gy^{|y|}g^{-1}=g1g^{-1}=gg^{-1}=1$ and so $x$ has order at most $|y|,$ in particular finite.

So to show the other direction, suppose $x\in G$ is not conjugate to an element of $G_1$ or $G_2.$  We will show that $x$ has infinite order.
\\
Every element of length 0 or 1 is trivially conjugate to itself, an element of either $G_1$ or $G_2,$ so the length of $x$ must be at least 2.
\\
Say the length of $x$ is 3.  Then $x$ is conjugate to an element of shorter length, but is not conjugate to an element of length 1 or 0, which would be a violation of our premise.  Then $x$ is conjugate to an element of length 2.
\\
Now say that we have shown that every element of $G$ with odd length less than $2n+1$ which is not conjugate to an element of $G_1$ or $G_2$ is conjugate to an element of even length at least 2.
\\
Then suppose the length of $x$ is $2n+1.$  We showed in the last part of the exercise that $x$ must be conjugate to an element $y$ of shorter length.  Either this element is of even length or it is of odd length less than $2n+1.$  If $y$ has length 0 then $y=1$ and $x$ is conjugate to 1, a contradiction.  So if $y$ is of even length, it is of length at least 2.
\\
Now suppose $y$ has odd length.  If $y$ is conjugate to an element of $G_1$ or $G_2$ then $x$ is as well since being a conjugate is an equivalence relation.  So $y$ is not conjugate to an element of $G_1$ or $G_2,$ so by our inductive premise is conjugate to an element of even length at least 2.  Then because being a conjugate is an equivalence relation, $x$ is also conjugate to an element of even length at least 2.
\\
This establishes inductively that every odd length element of $G$ which is not is conjugate to an element of even length at least 2.
\\
If $x$ has even length at least 2, then $x$ is the conjugate of itself trivially, so is a conjugate of an element of even length at least 2.
\\
Then every element of $G$ not conjugate to an element of $G_1$ or $G_2$ is conjugate to an element of even length at least 2.  Call such an element $y.$  So $gxg^{-1}=y$ for some $g\in G.$  Now suppose $x$ has finite order.  Then $y^{|x|}=gx^{|x|}g^{-1}=g1g^{-1}=1$ and $y$ has order at most $|x|,$ in particular finite.  This contradicts the first part of the exercise so $x$ does not have finite order, as desired.

425.
\\3. Suppose $G=G_1*G_2,$ where $G_1$ and $G_2$ are cyclic of orders $m$ and $n,$ respectively.  We show $m$ and $n$ are uniquely determined by $G.$
\\
a. Let $g$ and $h$ be generators for $G_1$ and $G_2,$ respectively.  We show that $G/[G,G]$ has order $mn.$
\\
Suppose $x$ is a word in $G$ of length at least 3.  Without loss of generality, then, $x=(g^{n_1},h^{n_2},g^{n_3},\ldots).$  Then if $y\in [G,G]$ then $x[G,G]=[G,G]x=[G,G]yx=yx[G,G].$  Let $y=(h^{-n_2},g^{-n_1},h^{n_2},g^{n_1}).$  Then $yx=(h^{-n_2},g^{n_3-n_1},\ldots)$ is in the same coset of $[G,G]$ as $x$ and has a strictly lower length.  Then every coset has a representative whose length is at most 2.  If such a representative is of form $x=(h^k,g^l)$ then multiplying by the commutator $(g^l,h^k,g^{-l},h^{-k})$ yields $g^lh^k$ in the same coset as $x.$  So each coset has a representative of form $(g^l,h^k)$ if $l$ and $k$ are allowed to range between $0$ and $m-1$ and $0$ and $n-1$ respectively.  Since there are precisely $m$ legitimate values for $l$ and $n$ legitimate values for $k,$ there can be at most $mn$ cosets of $[G,G],$ and thus the order of $G/[G,G]$ is at most $mn.$

Now say $0\le l_1,l_2<m;$ let $0\le k_1,k_2<n.$  Say $g^{l_1}h^{k_1}[G,G]=g^{l_2}h^{k_2}[G,G].$  Then $h^{-k_2}g^{-l_2}g^{l_1}h^{k_1}\in [G,G]$ so that its conjugate $g^{l_1-l_2}h^{k_1-k_2}$ is also in $[G,G].$  Then this is the product of commutators in $G.$
\\
The inverse of an element $g^l$ in $G_1$ is $g^{-l};$ the inverse of an element $h^k$ in $G_2$ is $h^{-k}.$  A word is the product of such elements; the inverse of a word is the product of the inverses in reverse order.
\\
The defined word operations $(\ldots,g^i,g^j,\ldots)\to (\ldots,g^{i+j},\ldots)$ and $(\ldots,g^i,g^{-i},\ldots)\to (\ldots,\ldots)$ leave fixed the total exponent of $g$ in a word; likewise for $h.$  Then the total exponents of $g$ and $h$ are invariants of an element of $G$ and do not change with different word representations.
\\
Then a commutator, which is the product of two words and their inverses, must have total exponents of both $h$ and $g$ equal to 0.  Then the product of commutators must have the same property, since evidently by inspection the total exponent of a product is the sum of the total exponents of the factors.
\\
Then if $g^{l_1-l_2}h^{k_1-k_2}\in[G,G]$ then $l_1-l_2$ and $k_1-k_2$ must both be 0, so that $l_1=l_2$ and $k_1=k_2.$
\\
Then each set of distinct legitimate values of $l$ and $k$ in the above expression determines a different coset of $[G,G],$ so that the order of $G/[G,G]$ is at least $mn.$  This concludes this portion of the proof.

b. Now we show that the largest integer such that $G$ has an element with order that integer is the greater of $m$ and $n.$
\\
First of all, if either $G_1$ or $G_2$ is trivial (suppose it to be $G_1$ without loss of generality) then any reduced word in $G$ can have only letters from $G_2,$ since the only letter from $G_1$ is 1 which reduces out of any word.  Then a reduced word must be of length 0 or 1 since adjacent elements of $G_2$ will be combined.  Then an element of $G$ is an element of $G_2.$  Since $G_2$ is a subgroup of $G$ they are the same group, and thus have the same order, $mn=1n=n.$
\\
So suppose both $G_1$ and $G_2$ are nontrivial.  Then the last exercise shows that the only elements of finite order in $G$ are the conjugates of the elements of $G_1$ and $G_2.$
\\
We showed above that a conjugate $x$ of an element $y$ with finite order had finite order at most $|y|.$  So $|x|\le |y|.$  Then $y$ is a conjugate of $x$ so $|y|\le |x|$ and they are equal.  So the highest finite order in $G$ is the highest finite order in $G_1\cup G_2.$  By definition, $|g|=m,|h|=n.$  If an element of $G_1$ had order greater than $m,$ it would generate a group of order greater than $m$ which could not be contained in $G_1,$ a contradiction.  For the same reason, $n$ is the maximum order for an element of $G_2.$  So any element of finite order in $G$ must have order at most $\max\{m,n\}.$

c. Now we show that $m$ and $n$ are uniquely determined by $G.$
\\
$[G,G]$ is uniquely determined by $G$ as $\{xyx^{-1}y^{-1}:x,y\in G\}$ and so $K=|G/[G,G]|$ is determined by $G.$  The set $S$ of finite order elements in $G$ is determined by $G$ as is the maximal order of its elements $N.$  Then $N,K/N$ are $m$ and $n$ in decreasing order.

438.
\\1. Let $X$ be the union of $S_1,\ldots, S_n,$ each of which is homeomorphic to $S^1,$ and that there is a point $p\in X$ with $S_i\cap S_j=\{p\}$ for $i\ne j.$
\\a. We show $X$ is Hausdorff if and only if each space $S_i$ is closed in $X.$
\\Suppose $X$ is Hausdorff.  Then for $1\le i\le n,\ S_i$ is homeomorphic to $S^1$ by the map $h:S^1\to S_i.$  Expand the range of $h$ to $X;$ then $h|X$ is continuous.  Its image is $S_i,$ which is compact since $S^1$ is compact.  A compact subspace of a Hausdorff space is closed so $S_i$ is closed in $X.$
\\To show the other way, suppose each $S_i$ is closed in $X.$  We show $X$ is Hausdorff.
\\
First we show that $p$ and any other point $x$ in $X$ have disjoint neighborhoods.
\\
We know $x$ is in $S_k$ for some $k.$  In $S_k,$ there are disjoint neighborhoods $U$ and $V$ of $x$ and $p.$  Then $S_k-U$ is a closed subset of $S_k$ containing $p$ and $S_k-V$ is a closed subset of $S_k$ containing $x.$  Then $S_k-V$ and $S_k-U$ are closed subsets of a closed subset ($S_k$) of $X$ and thus are closed in $X.$  So $S_k-U\bigcup_{i\ne k}S_i$ is the finite union of closed sets, therefore closed in $X.$  Then its complement $U'$ is open.  This complement is in fact $U$ (which contains $x$) as may be seen by inspection.
The complement of $S_k-V$ is also open.  This is $V'=V\bigcup_{i\ne k}S_i.$  It contains $p$ and its intersection with $U'$ is $V\cap U=\emptyset.$  So $V'$ and $U'$ are disjoint neighborhoods of $p$ and $x.$
\\
Note that if $x$ and $y$ are in different $S_k$ and neither is $p$ it follows from this last that they have disjoint neighborhoods, since $V'$ and $U'$ are disjoint neighborhoods of them by construction.
\\
Now, if $x$ and $y$ are in the same $S_k$ and neither is $p,$ we can generate neighborhoods $U_x$ and $U_y$ of $x$ and $y$ that are open in both $S_k$ and $X$ and are disjoint from $p.$  Then $U_x\cup U_y$ is open in both $S_k$ and $X.$  As a subset of the Hausdorff space $S_k$ it contains disjoint neighborhoods of $x$ and $y.$  Since these are open subsets of the open subset $U_x\cup U_y$ of $X,$ they are open in $X.$

b. We show that $X$ is Hausdorff if and only if the topology of $X$ is coherent with the subspaces $S_i.$
\\
Suppose $X$ is Hausdorff.  Then each $S_i$ is closed in $X.$  Let $C\cap S_i$ be closed in $S_i$ for each $1\le i\le n.$  Then $C\cap S_i$ is closed in $X.$  Since $C=\bigcup_{1\le i\le n}C\cap S_i$ is the finite union of closed sets it is closed in $X.$

To show the other direction, suppose $X$ is coherent with the subspaces $S_i.$  Then fix some $1\le k\le n.$  For $i\ne k,\ S_k\cap S_i=\{p\},$ which is closed in $S_i$ since one point sets are closed in its homeomorphic image $S^1.$  For $i=k,\ S_k\cap S_i=S_k=S_i,$ which is trivially closed in $S_k=S_i.$  So $S_k\cap S_i$ is closed in $S_i$ for all $1\le i\le n,$ so that $S_k$ is closed in $X.$  This is true for arbitrary $k$ so each $S_i$ is closed in $X.$  Then the last part of the exercise shows $X$ to be Hausdorff.

c. We give an example where $X$ is not Hausdorff.
Let $X$ be the union of two spaces $Y$ and $Z$ identified with the unit circle with a point $p$ of intersection identified as $(0,1)$ on both circles.  Let the points $q_1$ and $q_2$ be the points identified as $(1,0)$ on the respective circles.  As a basis for a topology on $X,$ let $\mathscr{B}$ consist of all open sets of the unit circle not containing $(0,1)$ or $(1,0)$ in either $Y$ or $Z$ as well as the set of unions $U\cup V$ of neighborhoods of $(0,1)$ in $Y$ and $Z$ respectively not containing $(1,0)$ and sets of form $U\cup V-q_1$ and $U\cup V-q_2$ for $U$ and $V$ respectively neighborhoods of $q_1$ and $q_2$ and not containing $p.$
\\
Then the subspace topology on $Y$ has as a basis the open sets of the unit circle not containing $p$ or $q_1,$ sets of form $U\cup V\cap Y=U$ where $U$ is a neighborhood of $p$ not containing $q_1,$ sets of form $U\cup V-q_1\cap Y=U\cap(Y-q_1)$ for $U$ a neighborhood of $q_1$ not containing $p,$ and sets of form $U\cup V-q_2\cap Y=U$ where $U$ is a neighborhood of $q_1$ not containing $p.$
\\
All of these are open in the standard topology of the unit circle.  The third type is an intersection of open sets; the other three are a priori open in $S^1.$
\\
Since the intersection of $1/n$-balls in $\mathbb{R}^2$ centered at each point of $S^1$ for $n\ge 1$ is a basis for the topology of $S^1$ that is contained in this basis, the subspace topology on $Y$ from $X$ is that of the unit circle.  The same is true for $Z.$
\\
But in $X,$ any neighborhood of $q_1$ must contain a basis element of form $U\cup V-q_2$ for $V$ a neighborhood of $q_2$ in the unit circle.  Any neighborhood of $q_2$ must contain a basis element of form $U'\cup V'-q_2$ for $V'$ a neighborhood of $q_2$ in the unit circle.  Then $V$ and $V'$ must contain the intersection of some $\epsilon>0$ and $\delta>0$ balls centered at $q_2$ in $\mathbb{R}^2,$ respectively, with $S^1.$  These balls cannot have intersection only at $q_2,$ since their intersection is a neighborhood of $q_2$ in the unit circle, so the neighborhoods of $q_1$ and $q_2$ are not disjoint and $X$ is not Hausdorff.

2. We suppose $X$ is a space that is the union of the closed subspaces $X_1,\ldots, X_n,$ and that there is a point $p$ such that $X_i\cap X_j=\{p\}$ for $i\ne j.$  Then $X$ is the wedge of $X_1,\ldots, X_n.$  We show that if for each $i,\ p$ is a deformation retract of an open set $W_i$ of $X_i,$ then $\pi_1(X,p)$ is the external free products of the groups $\pi_1(X_i,p)$ relative to the monomorphism induced by inclusion.
\\
To avoid trivialities, we assume also that the $X_n$ are path connected.
\\
This is trivially true for the case $n=1.$  Now suppose it is true up through $n-1.$
\\
$X_i-W_i$ is closed in $X_i,$ which is closed in $X,$ so its complement $W_i\cup \bigcup_{j\ne i} X_j$ is open in $X.$  Then let $U=W_n\cup\bigcup_{1\le j\le n-1} X_j$ and $V=\bigcap_{1\le i\le n-1}(W_i\cup \bigcup_{j\ne i} X_j)=X_n\cup\bigcup_{1\le j\le n-1} W_j.$  This is the finite intersection of open sets, therefore open.
\\
We have $U\cup V=X$ since $U$ contains $X_i$ for every index but $n$ and $V$ contains $X_n.$  We have also $U\cap V=\bigcup_{1\le j\le n} W_n.$  Call this set $W.$
\\
Let $f_i$ be a deformation retraction of $W_i$ onto $p.$  Then $f_i:W_i\times I\to W_i$ is continuous and fixes $p\times t.$  Let $g_i:X_i\times I\to X_i$ be the identity homotopy on $X_i.$
\\
The intersection of the domains of the $f_i$ and $g_i$ for differing $i$ is $\{p\}\times I$ which is the product of closed sets, therefore closed in $X\times I$ ($\{p\}$ is closed because it is the intersection of the closed sets $X_1\cap X_2$).  Then the pasting lemma tells us that $f_n\cup\bigcup_{1\le j\le n-1}g_n:U\times I\to U,$ $g_n\cup\bigcup_{1\le j\le n-1}f_n:V\times I\to V,$ and $\bigcup_{1\le j\le n}f_j:W\times I\to W$ are continuous.  Then they are deformation retractions from $U$ to $\bigcup_{1\le j\le n-1} X_j$ in the first case, from $V$ to $X_n$ in the second case, and from $W$ to $\{p\}$ in the third case.
\\
So $\pi_1(U,p)=\pi_1(\bigcup_{1\le j\le n-1} X_j,p)=\pi_1(X_1,p)*\cdots*\pi_1(X_{n-1},p)$ by our inductive premise.  $\pi_1(V,p)=\pi_1(X_n,p).$  And $\pi_1(W,p)$ is trivial.
\\
Then by the Seifert-Van Kampen corrolary, since $\pi_1(W,p)$ is trivial, there is an isomorphism between $\pi_1(U,p)*\pi_1(V,p)$ and $\pi_1(X,p).$  So $\pi_1(X,p)$ is isomorphic to $\pi_1(X_1,p)*\cdots*\pi_1(X_n,p)$ to validate the inductive premise for the next index.
\\
Then this identity is valid for all $n,$ as desired.

5. Let $S_n$ be the circle of radius $n$ in $\mathbb{R}^2$ whose center is at $(n,0).$  Let $Y$ be the union of all the $S_n,$ and let $p$ be their common point.

a. We show that $Y$ is not homeomorphic to the countably infinite wedge of circles $X$ or the infinite earring $Z.$

We show $X$ is not first countable.  Since $Y$ is a subspace of the first-countable space $\mathbb{R}^2,$ it is first-countable and so not homeomorphic to $X.$
\\
Characterize $X$ as the union of the sets $\{X_n\}$ each of which is homeomorphic to $S^1,$ the pairwise intersection of any two of which is the single point $\{x\},$ and where the topology of $X$ is coherent with the topologies of $X_n.$

Suppose $\{B_n\}$ is a countable basis for the shared point $x$ in $X.$  We show a contradiction.
\\
$B_n\cap X_n$ is a neighborhood of $x$ in $X_n.$  Then by the regularity of $X_n,\ x$ and $X_n-(B_n\cap X_n)$ have disjoint neighborhoods $U_n$ and $V_n.$  In particular, $U_n$ is a neighborhood of $x$ in $X_n$ properly contained in $B_n\cap X_n.$
\\
Then let $U=\bigcup U_n.$  This contains $x.$ We have $U\cap X_n=U_n.$  Since the topology of $X$ is coherent with the topology of the $X_n$ and the $U\cap X_n$ is open in $X_n,$ we know $U$ is open in $X.$  Then since $B_n\cap X_n\notin U\cap X_n$ this set contains no element of the local basis $\{B_n\},$ a contradiction.  So $X$ is not first countable.

We show $Z$ is compact.  Since $Y$ is a nonbounded subspace of $\mathbb{R}^2$ it is not compact so is not homeomorphic to $Z.$
\\
Let $\mathscr{B}$ be an open cover of $Z.$  Then it must contain some neighborhood $U$ of the origin $p.$  This $U$ must contain the intersection of some basis $\epsilon$-ball $B$ in $\mathbb{R}^2$ with $Z.$  For $n>2/\epsilon,$ the circle $C_n$ of radius $1/n$ centered at $(1/n,0)$ is contained entirely in $B.$  Then $\mathscr{B}$ is an open cover of $\bigcup_{n\le 2/\epsilon} C_n.$  This is a finite union of closed sets, so closed.  As a subspace of the radius 2 ball centered at the origin, this closed set is bounded in $\mathbb{R}^2$ and therefore compact.  Then there is a finite subcover $\mathscr{A}.$ of $\bigcup_{n\le 2/\epsilon} C_n$ in $\mathscr{B}.$  Then $\mathscr{A}\cup\{U\}$ is a finite subcover of $Z,$ so that $Z$ is compact as well.

b. We show that $\pi_1(Y,p)$ is a free group with $\{[f_n]\}$ as a system of free generators, where $f_n$ is a loop representing a generator of $\pi_1(S_n, p).$
\\
Define $F_+:\mathbb{R}^2\times I\to \mathbb{R}^2$ as $F_+(x,y,t)=(1-t)x\times\sqrt{(t-t^2)x^2+(1-t)y^2}.$  Define $F_-:W\times I\to W$ as $F_-(x,y,t)=\pi_1\circ F_+\times -\pi_2 \circ F_+(x,y,t).$
Since $\sqrt{\ }$ is continuous on the nonnegative numbers, and $t-t^2,1-t,x^2,$ and $y^2$ are all nonnegative on the domain of the function, both of these have projections which are the compositions and arithmetic combinations of real-valued continuous projections and constant functions.  Then they are continuous.
\\
Suppose $x>0$ and $y\ne 0.$  Then
$$
d(F_{\pm}(x,y,t),\frac{x^2+y^2}{2x}\times 0)=\sqrt{((1-t)x-\frac{x^2+y^2}{2x})^2+(t-t^2)x^2+(1-t)y^2}
$$$$
=\sqrt{(1-t)^2x^2-(1-t)(x^2+y^2)+(\frac{x^2+y^2}{2x})^2+(t-t^2)x^2+(1-t)y^2}
=\frac{x^2+y^2}{2x}
$$
\\
If $x=0,y=0,$ then $F_{\pm}(x,y,t)=0\times \pm\sqrt{0+0}=(0,0).$
\\
So if $(x,y)\in S_n$ but is not the point $2n\times 0,$ then $F_{\pm}(x,y,t)\in S_n.$
\\
$F_{\pm}(x,y,0)=x\times \pm |y|.$
$F_{\pm}(x,y,1)=0\times 0$ so $F_{\pm}$ is a deformation retraction of $\mathbb{R}_+\times \mathbb{R}_{\pm}+\{0\times 0\}$ onto $p.$  Then since the intersection of these two sets (that is, the space $\mathbb{R}_+\times \mathbb{R}\cup\{0\times 0\}-0\times\mathbb{R}_+.$ is the set $\{p\}$ on which they agree, and since they are each closed in their union, having open quarter-planes as the disjoint neighborhoods of their respective complements, their union $F=F_+\cup F_-$ is a deformation retraction of the half plane $x>0$ with the origin adjoined and without the positive $x$ axis onto the origin.

Now, if $f$ is a loop in $Y$ based at $p$ then $f([0,1])$ is the image of a compact set, therefore compact, therefore bounded, therefore contained in an $M$-ball  for some $M.$  Then $f([0,1])$ contains no point $x\times 0$ for any $x\ge 2M.$  So $f([0,1])$ is contained in the union of the wedge of $M$ circles and the union of all the other circles in $Y$ without the points $x\times 0.$  The wedge of $M$ circles (denoted $Y_1$)is closed in $\mathbb{R}^2$ so it is closed in $Y.$  The wedge of $M$ circles without the point $p$ is open in $Y$ because every point can be separated from $p$ by one neighborhood and then locally within that neighborhood from every other circle.  Then the complement of this subspace, the union of all the circles with radius at least $M+1$ (denoted $Y_2$) is closed in $Y.$  Then $Y_1\cup Y_2=Y$ and $Y_1\cap Y_2=\{p\}.$

Let $I_i=f^{-1}(Y_i)$ for $i\in\{1,2\}.$  Then $I_i$ is the preimage of a closed set, hence closed.  $I_1\cup I_2=I$ and $I_1\cap I_2=f^{-1}(p).$
\\
Define $g_1:I_1\times I\to Y$ as $g_1(t,s)=f(t)$ and $g_2:I_2\times I\to Y$ as $g_2(t,s)=F(f(t),s).$  Then $g_1(f^{-1}(p),s)=f(f^{-1}(p))=p=F(p,s)=F(f(f^{-1}(p)),s)=g_2(f^{-1}(p),s).$
\\
So $g_1$ and $g_2$ agree on the intersection of their relative domains and thus $g=g_1\cup g_2:I^2\to Y$ is continuous.  It satisfies $g(t,0)$ is either $g_1(t,0)=f(t)$ or $g_2(t,0)=F(f(t),0)=f(t).$  Also $g(0,s)$ is either $g_1(0,s)=f(0)=p$ or $g_1(0,s)=F(f(0),s)=F(p,s)=p,$ and likewise for $g(1,s).$  Then $g$ is a path homotopy between $f$ and $f'=g(t,1).$
\\
Since $g(I_1,1)\subset f(I_1)$ and $g(I_2,1)=p\subset f(I_1)$  We know $f'$ is a path in $f(I_1)=Y_1,$ the wedge of $M$ circles.

Then this path $f'$ is the product of generators of the fundamental group of the wedge of $M$ circles, that is, of elements of the form $[f_n]$ for any $1\le n\le M.$
\\
So the fundamental group of $Y$ is generated by $\{[f_n]\}$ subject, perhaps, to some relations.  But if $f$ is path-homotopic to a constant loop in $Y$ then by the above it is path homotopic to some $f'$ which is a loop in a finite wedge of circles, so that any relation would also apply to the free group which is the fundamental group of this wedge.  Since the wedge has no relations, neither does $\pi_1(Y,p)$ so that it is a free group, as desired.

445.
\\1. We find spaces with fundamental groups isomorphic to given groups.  Let $X_i$ denote the $i$-fold dunce cap.
\\a. $\mathbb{Z}/n\times \mathbb{Z}/m.$
\\Since the $k$-fold dunce cap has fundamental group isomorphic to $\mathbb{Z}/k$ and the fundamental group of a product (of two spaces) is isomorphic to the direct product of the fundamental groups of the two factor spaces, this is the fundamental group of $X_n\times X_m.$

b. $\mathbb{Z}/n_1\times\cdots\times \mathbb{Z}/n_k.$
\\Suppose that the fundamental group of $X=X_{n_1}\times \cdots\times X_{n_i}$ is isomorphic to $\mathbb{Z}/n_1\times\cdots\times \mathbb{Z}/n_i.$  This is certainly true for $i=1.$  Then the fundamental group of the product space $X\times X_{n_{i+1}}$ is isomorphic to the direct product of the fundamental group of $X$ with the fundamental group of $X_{n_{i+1}},$ that is, $\mathbb{Z}/n_1\times\cdots\times \mathbb{Z}/n_{i+1}.$  So the inductive statement is valid for all $i$ for which it is defined and the fundamental group of $X_{n_1}\times\cdots\times X_{n_k}$ is isomorphic to the desired group.

c. $\mathbb{Z}/n *\mathbb{Z}/m$
\\
$X_i$ is constructed as a quotient of $B^2$ via the map $q_i.$  Then since $q_i$ takes the closed set $S^1$ to a closed set $S_i,$ we have $q_i^{-1}(X_i-S_i)$ is the inverse image of an open set, therefore open.  Since no point of the interior and every point of the boundary of $B^2$ is taken to $S_i,$ this is just the image of the interior of $B^2.$
\\
On this subspace of $B^2,$ the straight line homotopy from the identity map to the constant map at the origin is a deformation retraction to the origin. So $X_i-S_i$ is an open set containing $q_i(0\times 0)$ which deformation retracts to $q_i(0\times 0)$ in $X_i$ by the induced map.

Let $X_0$ be the disjoint union of the spaces $X_n,X_m$ with the topologies of $X_n$ and $X_m$ as a basis.  Let $X$ be the quotient space generated by identifying $q_n(0\times 0)$ in $X_n$ with $q_m(0\times 0)$ in $X_m.$  Let $q$ be the quotient map.  A closed set $C$ is $X_0$ is equal to the union of the closed sets $X_n\cap C$ and $X_m\cap C.$  Then $q^{-1}(q(C))=(X_m\cap C)\cup (X_n\cap C)\cup((q_n(0\times0)\cup q_m(0\times 0))\cap q^{-1}(q(C))).$  This is the finite union of closed sets, therefore closed.  Then $q(C)$ is closed so that $q$ is closed.
\\
Then $q(X_m)$ and $q(X_n)$ are closed in $X$ and since $q$ is perfect, $X$ is Hausdorff.  $q(X_m)$ is homeomorphic to $X_m$ because $q|X_m$ with its range restricted is a bijective continuous map from a compact space into a Hausdorff space.
\\
So by the result from exercise 2 in the last section, the fundamental group of $X$ is isomorphic to the external free product of the fundamental groups of $X_n$ and $X_m,$ that being the desired group.

d. $\mathbb{Z}/n_1 *\cdots*\mathbb{Z}/n_k.$
\\The logic of this case follows closely the logic of the last case.  Assume the same terminology.

Let $X_0$ be the disjoint union of the spaces $X_{n_i}$ for $1\le i\le k$ with the topologies of $X_{n_i}$ as a basis.  Let $X$ be the quotient space generated by identifying $q_{n_i}(0\times 0).$  Let $q$ be the quotient map.  A closed set $C$ is $X_0$ is equal to the $\bigcup_{1\le i\le k}(X_{n_i}\cap C).$  Then $q^{-1}(q(C))=\bigcup_{1\le i\le k}(X_{n_i}\cap C)\cup ((\bigcup_{1\le i\le k}q_{n_i}(0\times 0))\cap q^{-1}(q(C))).$  This is the finite union of closed sets (the last set in the union is the union of one point sets), therefore closed.  Then $q(C)$ is closed so that $q$ is closed.
\\
Then $q(X_{n_i})$ is closed in $X$ and since $q$ is perfect, $X$ is Hausdorff.  $q(X_{n_i})$ is homeomorphic to $X_{n_i}$ because $q|X_{n_i}$ with its range restricted is a bijective continuous map from a compact space into a Hausdorff space.
\\
So by the result from exercise 2 in the last section, the fundamental group of $X$ is isomorphic to the external free product of the fundamental groups of the $X_{n_i}$ that being the desired group.


\end{document}

